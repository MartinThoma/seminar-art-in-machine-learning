\documentclass[technote,a4paper,leqno]{IEEEtran}
\pdfoutput=1
\usepackage{amssymb, amsmath}

% \usepackage[a-1b]{pdfx}  % Not supported by arxiv
\usepackage{filecontents}
\begin{filecontents*}{\jobname.xmpdata}
    \Keywords{recognition\sep machine learning\sep neural networks\sep symbols\sep multilayer perceptron}
    \Title{Art in Machine Learning}
    \Author{Martin Thoma}
    \Org{}
    \Doi{}
\end{filecontents*}

\RequirePackage{ifpdf}
\ifpdf \PassOptionsToPackage{pdfpagelabels}{hyperref} \fi
\RequirePackage{hyperref}
\usepackage{parskip}
\usepackage[pdftex,final]{graphicx}
\usepackage{csquotes}
\usepackage{bm}
\usepackage{blindtext}\usepackage{braket}
\usepackage{booktabs}
\usepackage{multirow}
\usepackage{pgfplots}
\pgfplotsset{compat=newest}
\usepackage{wasysym}
\usepackage[symbol]{footmisc}
\usepackage{float}
\usepackage{caption}
\usepackage{subcaption}
\makeatletter
\newcommand\mynobreakpar{\par\nobreak\@afterheading}
\makeatother
\usepackage[noadjust]{cite}

\usepackage[nameinlink, noabbrev, capitalise]{cleveref} % has to be after hyperref, nthe l.137 ...eam attr{/N 4}  file{sRGBIEC1966-2.1.icmorem, amsthm
\usepackage[binary-units,group-separator={,}]{siunitx}
\sisetup{per-mode=fraction,binary-units=true}
\DeclareSIUnit\pixel{px}
\usepackage{glossaries}
\loadglsentries[main]{glossary}
\makeglossaries

\title{Pixel-wise Segmentation of Street with Neural Networks}
\author{%
    \IEEEauthorblockN{Martin Thoma}\\
    \IEEEauthorblockA{E-Mail: info@martin-thoma.de} % ORCID: http://orcid.org/0000-0002-6517-1690
}

\hypersetup{
    pdfauthor   = {Martin Thoma},
    pdfkeywords = {recognition, machine learning, neural networks, classification, multilayer perceptron, deep learning},
    pdfsubject  = {Recognition},
    pdftitle    = {Art in Machine Learning},
}


\crefname{table}{Table}{Tables}
\crefname{figure}{Figure}{Figures}

\usepackage{microtype}

\begin{document}
\maketitle
\begin{abstract}
Recent machine learning techniques can be modified to produce results which
look interesting. Those results did not exist before; it is not a trivial
combination of the data which was fed into the machine learning system. The
obtained results come in multiple forms: As images, as text and as audio.

This paper gives a high level overview of how they are created and gives some
examples. It is ment to be a summary of the current work.
\end{abstract}

%!TEX root = art-in-machine-learning.tex

\section{Introduction}
\label{sec:introduction}
According to \cite{LongmanDCE06} \textit{creativity} is \enquote{the ability to
use your imagination to produce new ideas, make things etc.} and
\textit{imagination} is \enquote{the ability to form pictures or ideas in your
mind}.

Recent advances in machine learning produce results which the author would
intuitively call creative. A high-level overview over several of those
algorithms are described in the following.

This paper is structured as follows: \Cref{sec:ml-basics} introduces the
reader on a very simple and superficial level to machine learning,
\cref{sec:images} gives examples of art with images, \cref{sec:text-generation}
gives examples of machines producing textual content, and \cref{sec:music}
gives examples of machine learning and music. A discussion follows in
\cref{sec:discussion}.

%!TEX root = art-in-machine-learning.tex

\section{Basics of Machine Learning}
\label{sec:ml-basics}
The traditional approach of solving problems with software is to program
machines to do so. The task is divided in as simple sub-tasks as possible,
the subtasks are analyzed and the machine is instructed to process the input
with human-designed algorithms to produce the desired output. However, for
some tasks like object recognition this approach is not feasable. There are
way to many different objects, different lightnig situations, variations in
rotation and the arrangement of a scene for a human to think of all of them and
model them. But with the internet, cheap computers, cameras, crowd-sourcing
platforms like Wikipedia and lots of Websites, services like Amazon Mechanical
Turk and several other changes in the past decades a lot of data has become
available. The idea of machine learning is to make use of this data.

A formal definition of the field of Machine Learning is given by
Tom~Mitchel~\cite{Mitchell97}:
\begin{displayquote}
A computer program is said to learn from experience~$E$ with respect to some
class of tasks~$T$ and performance measure~$P$, if its performance at tasks
in~$T$, as measured by~$P$, improves with experience $E$.
\end{displayquote}

This means that machine learning programs adjust internal parameters to fit the
data they are given. Those computer programs are still developed by software
developers, but the developer writes them in a way which makes it possible to
adjust them without having to re-program everything. Machine learning programs
should generally improve when they are fed with more data.

The field of machine learning is related to statistics. Some algorithms
directly try to find models which are based on distribution assumptions of the
developer, others are more general.

A common misunderstanding of people who are not related in this field is that
the developers don't understand what their machine learning program is doing.
It is understood very well in the sense that the developer, given only a pen,
lots of paper and a calculator could calculate the same result as the machine
does when he gets the same data. And lots of time, of course. It is not
understood in the sense that it is hard to make predictions how the algorithm
behaves without actually trying it. However, this is similar to expecting from
an electrical engineer to explain how a computer works. The electrical engineer
could probably get the knowledge he needs to do so, but the amount of time
required to understand such a complex system from basic building blocks is
a time-intensive and difficult task.

An important group of machine learning algorithms was inspired by biological
neurons and are thus called \textit{artifical neural networks}. Those networks
are based on mathematical functions called \textit{artifical neurons} which
take $n \in \mathbb{N}$ numbers $x_1, \dots, x_n \in \mathbb{R}$ as input,
multiply them with weights $w_1, \dots, w_n \in \mathbb{R}$, add them and apply
a so called \textit{activation function} $\varphi$. One example of such a
function is the sigmoid function $\varphi(x) = \frac{1}{1+e^{-x}}$. Those
functions act as building blocks for more complex systems as they can be
chained and grouped in layers. The interesting question is how the parameters
$w_i$ are learned. This is usually done by an optimization technique called
\textit{gradient descent}. The gradient descent algorithm takes a function
which has to be derivable, starts at any point of the surface of this error
function and makes a step in the direction which goes downwards. Hence it tries
to find a minimum of this high-dimensional function.

There is, of course, a lot more to say about machine learning. The interested
reader might want to read the introduction given by Mitchell~\cite{Mitchell97}.

%!TEX root = art-in-machine-learning.tex

\section{Image Data}%
\label{sec:images}%

\subsection{Google DeepDream}%
\label{subsec:google-deepdream}%
The gradient descent algorithm which optimizes most of the parameters in neural
networks is well-understood. However, the effect it has on the recognition
system is difficult to estimate. \cite{inceptionism2015} proposes a technique
to analyze the weights learned by such a network.

For example, consider a neural network which was trained to recognize various
images like bananas. This technique turns the network upside down and starts
with random noise. To analyze what the network considers bananas to look like,
the random noise image is gradually tweaked so that it generates the output
\enquote{banana}. Additionally, the changes can be restricted in a way that the
statistics of the input image have to be similar to natural images. One example
of this is that neighboring pixels are correlated.
\goodbreak
Another technique is to amplify the output of layers. This was described
in~\cite{inceptionism2015}:\nobreak%
\begin{displayquote}
We ask the network: \enquote{Whatever you see there, I want more of it!} This
creates a feedback loop: if a cloud looks a little bit like a bird, the network
will make it look more like a bird. This in turn will make the network
recognize the bird even more strongly on the next pass and so forth, until a
highly detailed bird appears, seemingly out of nowhere.
\end{displayquote}

The name \enquote{Inceptionism} comes from the science-fiction movie
\enquote{Inception}~(2010). One reason it might be chosen is because neural
networks are structured in layers. Recent publications tend to have more and
more layers. The used jargon is to say they get \enquote{deeper}. As this
technique as published by Google engineers, images generated by this are called
\textit{Google DeepDream}.

It has become famous in the internet. (TODO: Reddit)
Images and videos published by the Google engineers can be seen
at~\href{https://goo.gl/Bydofw}{https://goo.gl/Bydofw}. TODO: Add own images.


\subsection{Magic The Gathering card creation}

TODO: \href{http://nerdist.com/what-happens-when-artificial-intelligence-makes-magic-the-gathering-cards/}{source}


\subsection{Caption generation}
Generating captions for images can be seen as a creative task, too. Interesting
captions don't only describe what one can obvioulsy see on the image, but also
interpret the image to a certain degree.

TODO: Show example.


\subsection{Artistic Style Imitation}
In~\cite{gatys2015neural}: TODO

(TODO: Apply \href{http://gitxiv.com/posts/jG46ukGod8R7Rdtud/a-neural-algorithm-of-artistic-style}{http://gitxiv.com/posts/jG46ukGod8R7Rdtud/a-neural-algorithm-of-artistic-style} and add images)



TODO: \cite{shih2014style} \href{http://gitxiv.com/posts/eoxDf59kBz87tvkX3/style-transfer-for-headshot-portraits}{http://gitxiv.com/posts/eoxDf59kBz87tvkX3/style-transfer-for-headshot-portraits}

\subsection{Drawing Robots}
Patrick Tresset's Robots

\subsection{Others}

TODO: \href{http://www.thepaintingfool.com/}{http://www.thepaintingfool.com/}
%!TEX root = art-in-machine-learning.tex

\section{Text Generation}%
\label{sec:text-generation}%

\subsection{Similar Texts Generation}
TODO: \href{http://karpathy.github.io/2015/05/21/rnn-effectiveness/}{http://karpathy.github.io/2015/05/21/rnn-effectiveness/}

\subsection{Chatbots}%
\label{subsec:chatbots}%

TODO: \href{http://uk.businessinsider.com/google-tests-new-artificial-intelligence-chatbot-2015-6?IR=T}{http://uk.businessinsider.com/google-tests-new-artificial-intelligence-chatbot-2015-6?IR=T}

%!TEX root = art-in-machine-learning.tex

\section{Audio Data}
\label{sec:music}

Common machine learning tasks which involve audio data are speech recognition,
speaker identification, identification of songs. This leads to some
less-common, but interesting topics: The composition of music, the synthesizing
of audio as art. While the composition might be considered in
\cref{sec:text-generation}, we will now investigate the work which was done in
audio synthesization.


\subsection{Emily Howell}
David Cope created a project called \enquote{Experiments in Musical
Intelligence} (short: EMI or Emmy) in 1984~\cite{Cope1987}. He introduces the idea of
seeing music as a language which can be analyzed with natural language
processing (NLP) methods. Cope mentions that EMI was more useful to him, when
he used the system to \enquote{create small phrase-size textures as next
possibilities using its syntactic dictionary and rule base}~\cite{Cope1987}.

In 2003, Cope started a new project which was based on EMI: Emily
Howell~\cite{cope2013well}. This program is able to \enquote{creat[e] both
highly authentic replications and novel music compositions}. The reader might
want to listen to~\cite{Cope2012} to get an impression of the beauty of the
created music.

According to Cope, an essential part of music is \enquote{a set of instructions
for creating different, but highly related self-replications}. Emmy was
programmed to find this set of instructions. It tries to find the
\enquote{signature} of a composer, which Cope describes as \enquote{contiguous
patterns that recur in two or more works of the composer}.

The new feature of \textit{Emily Howell} compared to \textit{Emmy} is that
Emily Howell does not necessarily remain in a single, already known style.

Emily Howell makes use of association network. Cope emphasizes that this is not
a form of a neural network. However, it is not clear from~\cite{cope2013well}
how exactly an association network is trained. Cope mentions that Emily
Howell is explained in detail in~\cite{cope2005computer}.


\subsection{GRUV}

Recurrent neural networks --- \gls{LSTM} networks, to be exact --- are used
in~\cite{nayebigruv} together with \gls{GRU} to build a network which can be
trained to generate music. Instead of taking notes directly or MIDI files,
Nayebi and Vitelli took raw audio waveforms as input. Those audio waveforms are
feature vectors given for time steps $0, 1, \dots, t-1, t$. The network is
given those feature vectors $X_1, \dots, X_t$ and has to predict the following
feature vector $X_{t+1}$. This means it continues the music. As the input is
continuous, the problem was modeled as a regression task. \Gls{DFT} was used on
chunks of length $N$ of the music to obtain features in the frequency domain.

An implementation can be found at~\cite{gruvGitHub} and a demonstration can
be found at~\cite{Vitelli2015}.


\subsection{Audio Synthesization}
Audio synthesization is generating new audio files. This can either be music or
speech. With the techniques described before, neural networks can be trained to
generate music note by note. However, it is desirable to allow multiple notes
being played at the same time.

This idea and some others were applied by Daniel Johnson. He wrote a very good
introduction into neural networks for music composition which explains those
ideas~\cite{Johnson2015}. Example compositions are available there, too. He
also made the code for his Biaxial Recurrent Neural Network available
under~\cite{Johnson2015a}.

% see also: \cite{sarroff2014musical}
% http://gitxiv.com/posts/dQxgWraeWgvpKJXLG/musical-audio-synthesis-using-autoencoding-neural-nets

% Deep Belief Network
% \href{http://gitxiv.com/posts/kZz9PCRcktdYSrWnp/deephear-composing-and-harmonizing-music-with-neural}{http://gitxiv.com/posts/kZz9PCRcktdYSrWnp/deephear-composing-and-harmonizing-music-with-neural}


% \subsection{The Generative Electronica Research Project}

% \href{http://metacreation.net/gerp/}{http://metacreation.net/gerp/}
% TODO: \cite{Eigenfeldt2013}

%!TEX root = art-in-machine-learning.tex

\section{Discussion}
\label{sec:discussion}

What does these examples mean for our understanding of creativity? How does it
influence how much we value art? Could we define art and creativity better
after having those and similar results?


\bibliographystyle{IEEEtranSA}
\bibliography{art-in-machine-learning}
%\printglossary
\end{document}

%!TEX root = art-in-machine-learning.tex

\section{Introduction}
\label{sec:introduction}
According to \cite{LongmanDCE06} \textit{creativity} is \enquote{the ability to
use your imagination to produce new ideas, make things etc.} and
\textit{imagination} is \enquote{the ability to form pictures or ideas in your
mind}.

% <seminar>
In the science fiction literature and and movies there are many examples of
\glspl{AI}, of which a couple are creative: The movie
\textit{Chappie}~(2015), in which a robot starts from the intelectual stage of
a child is one of them. The humanoid robot Chappie quickly learns how to speak,
how to interact with humans and also how to draw paintings. However, Chappie
only draws very closely what he has seen. The movie
\textit{Ex Machina}~(2015) has a humanoid \gls{AI} which is able to not only
draw the environment as it percieves it, but portrays an impression of a
character. The novel \textit{The Android's Dream} by John Scalzi describes how
an~\gls{AI} is formed: A human is scanned, the complete structure of the brain
--- atom by atom, impulse by impulse --- is captured. After that, a physical
emulation of this brain is run. The \gls{AI} is described as if it were the
same person, except for the fact that he does not have a body anymore. Although
this character does not directly create any artistic content, it is not
far-fetched to say that he is able to do whatever he could do before.

In contrast, the \glspl{AI} pictured in \textit{Humans as Gods} by Sergey
Snegov makes not the impression of being creative. It is called a
\textit{computer} all the time. A very powerful computer which is able to
understand commands in natural language, make astonishing predictions even for
unknown situations, automatically translate new languages. The only sign of
creativity of \glspl{AI} in Snegov's book is music which gets composed directly
for the characters. As the \gls{AI} is connected to the body of the person ---
it is even able to read the persons mind from the beginning of the live of the
person to death --- it is possible that this music creation happens on the fly.
Getting directly feedback of a person and having that much information about
a person in combination if such a powerful reasoning system could mean that the
artificial composer only makes clever predictions, but does not form own, new
ideas as it is part of the definition of~\cite{LongmanDCE06}.

% </seminar>

Let's get back to the real world: Recent advances in machine learning produce
results which the author would intuitively call creative. A high-level overview
over several of those algorithms are described in the following.

This paper is structured as follows: \Cref{sec:ml-basics} introduces the
reader on a very simple and superficial level to machine learning,
\cref{sec:images} gives examples of creativity with images,
\cref{sec:text-generation} gives examples of machines producing textual
content, and \cref{sec:music} gives examples of machine learning and music. A
discussion follows in
\cref{sec:discussion}.

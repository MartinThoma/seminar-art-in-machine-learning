%!TEX root = art-in-machine-learning.tex

\section{Image Data}%
\label{sec:images}%

\subsection{Google DeepDream}%
\label{subsec:google-deepdream}%
The gradient descent algorithm which optimizes most of the parameters in neural
networks is well-understood. However, the effect it has on the recognition
system is difficult to estimate. \cite{inceptionism2015} proposes a technique
to analyze the weights learned by such a network.

For example, consider a neural network which was trained to recognize various
images like bananas. This technique turns the network upside down and starts
with random noise. To analyze what the network considers bananas to look like,
the random noise image is gradually tweaked so that it generates the output
\enquote{banana}. Additionally, the changes can be restricted in a way that the
statistics of the input image have to be similar to natural images. One example
of this is that neighboring pixels are correlated.
\goodbreak
Another technique is to amplify the output of layers. This was described
in~\cite{inceptionism2015}:\nobreak%
\begin{displayquote}
We ask the network: \enquote{Whatever you see there, I want more of it!} This
creates a feedback loop: if a cloud looks a little bit like a bird, the network
will make it look more like a bird. This in turn will make the network
recognize the bird even more strongly on the next pass and so forth, until a
highly detailed bird appears, seemingly out of nowhere.
\end{displayquote}

The name \enquote{Inceptionism} comes from the science-fiction movie
\enquote{Inception}~(2010). One reason it might be chosen is because neural
networks are structured in layers. Recent publications tend to have more and
more layers. The used jargon is to say they get \enquote{deeper}. As this
technique as published by Google engineers, images generated by this are called
\textit{Google DeepDream}.

It has become famous in the internet. (TODO: Reddit)
Images and videos published by the Google engineers can be seen
at~\href{https://goo.gl/Bydofw}{https://goo.gl/Bydofw}. TODO: Add own images.


\subsection{Magic The Gathering card creation}

TODO: \href{http://nerdist.com/what-happens-when-artificial-intelligence-makes-magic-the-gathering-cards/}{source}


\subsection{Caption generation}
Generating captions for images can be seen as a creative task, too. Interesting
captions don't only describe what one can obvioulsy see on the image, but also
interpret the image to a certain degree.

TODO: Show example.


\subsection{Artistic Style Imitation}
In~\cite{gatys2015neural}: TODO

(TODO: Apply \href{http://gitxiv.com/posts/jG46ukGod8R7Rdtud/a-neural-algorithm-of-artistic-style}{http://gitxiv.com/posts/jG46ukGod8R7Rdtud/a-neural-algorithm-of-artistic-style} and add images)



TODO: \cite{shih2014style} \href{http://gitxiv.com/posts/eoxDf59kBz87tvkX3/style-transfer-for-headshot-portraits}{http://gitxiv.com/posts/eoxDf59kBz87tvkX3/style-transfer-for-headshot-portraits}

\subsection{Drawing Robots}
Patrick Tresset's Robots

\subsection{Others}

TODO: \href{http://www.thepaintingfool.com/}{http://www.thepaintingfool.com/}